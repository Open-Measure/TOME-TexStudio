\newpage \subsection*{Orphan File } \subsection*{Alternative Forms } \begin{enumerate} \item  Orphan  \end{enumerate} \subsection*{Definitions } \begin{DIC_Def}{Definition 1Computer Science }{} \paragraph{} A file whose owner has been deleted.  \end{DIC_Def} \subsection*{Related Terms } \begin{enumerate} \item  File  \item  (link) \href{Orphan (Dictionary Entry) }{ }   \end{enumerate} \subsection*{Quotes } \begin{DIC_BlockQuote} Once you have removed a user, you may want to verify that the user's old UID no longer owns files on the system. To find the paths of orphaned files, you can use the find command with the -nouser argument. Because find has a way of ``escaping'' onto network servers if you're not careful, it's usually best to check filesystems individually with -xdev:\$ sudo find filesystem -xdev -nouser  \end{DIC_BlockQuote} (Nemeth et al., 2011, p. 198)  \paragraph{} (  (link) \href{Nemeth et al., 2011 }{ } , p. 198)  \subsection*{Bibliography } \begin{enumerate} \item  (link) \href{Nemeth et al., 2011 }{ }   \end{enumerate} \subsection*{See Also } 