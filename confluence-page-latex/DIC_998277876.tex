\newpage \subsection*{AWS IAM } \subsection*{Definitions } \begin{DIC_Def}{Definition 1AWS }{} \paragraph{} The native IAM platform in AWS.  \end{DIC_Def} \subsection*{Related Terms } \begin{enumerate} \item  AWS  \item  AWS Account  \item  (link) \href{AWS Account Root User (Dictionary Entry) }{ }   \item  (link) \href{AWS IAM Group (Dictionary Entry) }{ }   \item  (link) \href{AWS IAM Policy (Dictionary Entry) }{ }   \item  (link) \href{AWS IAM Role (Dictionary Entry) }{ }   \item  (link) \href{AWS IAM Temporary Security Credentials (Dictionary Entry) }{ }   \item  (link) \href{AWS IAM User (Dictionary Entry) }{ }   \end{enumerate} \subsection*{Quotes } \begin{DIC_BlockQuote} AWS Identity and Access Management (IAM) is a web service that helps you securely control access to AWS resources. You use IAM to control who is authenticated (signed in) and authorized (has permissions) to use resources.  \end{DIC_BlockQuote} (AWS, 11/2020, p. 1)\\ (Online: \url{https://docs.aws.amazon.com/IAM/latest/UserGuide/introduction.html})  \paragraph{} (  (link) \href{AWS, 11/2020 }{ } , p. 1)  (Online:  https://docs.aws.amazon.com/IAM/latest/UserGuide/introduction.html  )  \subsection*{Bibliography } \begin{enumerate} \item  (link) \href{AWS, 11/2020 }{ }   \end{enumerate} \subsection*{See Also } label = "aws-iam"  false  title  