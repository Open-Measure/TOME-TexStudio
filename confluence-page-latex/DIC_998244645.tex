\section*{Stability of Access Decision Factors }\subsection*{Definitions } \begin{DIC_Def}{Definition 1 }{}The average period during which access decision factors are only subject to slight disturbance, prolonging the validity of previously defined access permissions. A disturbance of access decision factors beyond some threshold triggers the requirement to adapt access permissions. Distinct access control methods (e.g. ACL, RBAC, ABAC, PBAC) are varyingly efficient in the way they enable modifications of access permissions. \end{DIC_Def}\subsection*{Related Terms }\begin{itemize} \item   ABAC \item   Access (Dictionary Entry) \item   Access Control (Dictionary Entry) \item   Access Control List (Dictionary Entry) \item   PBAC \item   RBAC \end{itemize} \subsection*{Quotes }\begin{quote} 3.1 \emph{Stability of Access Decision Factors} -- When the basis for access decisions is relatively stable, use of mechanisms such as ACLs lends itself more readily. Administrative processes typically required to maintain these lists are time-intensive and not particularly well suited to situations where significant changes and updates are required frequently. On the other hand, use of a flexible Attribute Management enterprise service where attributes can be easily managed, may be more responsive and thus, more operationally effective. \end{quote}  ( Farroha and Farroha, 2012 , p. 3) \subsection*{Bibliography }\begin{itemize} \item   Farroha and Farroha, 2012 \end{itemize} \subsection*{See Also }false title label = "stability-of-access-decision-factors" 