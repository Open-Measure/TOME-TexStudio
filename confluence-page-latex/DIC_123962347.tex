\newpage \subsection*{Access } \subsection*{Alternative Forms } \begin{enumerate} \item  Acc.  \texttt{ Abbreviation  }   \end{enumerate} \subsection*{Definitions } \begin{DIC_Def}{Definition 1Noun, ability }{} \paragraph{} The ability to communicate with or operate discrete functions of a system.  \end{DIC_Def} \begin{DIC_Def}{Definition 2Noun, event }{} \paragraph{} The corresponding event when this ability is used.  \end{DIC_Def} \begin{DIC_Def}{Definition 3Verb }{} \paragraph{} The action of using this ability.  \end{DIC_Def} \subsubsection*{Note } \paragraph{} The term  \emph{ access  } is ambiguous as it may designate both an ability to do something or the event that results from using that ability. Speakers should pay attention to this ambiguity and assure that the context makes it clear. When the  \emph{ access  } ability is meant, the expression  \emph{ to have access to  } is less ambiguous.  \subsection*{Related Terms } \begin{enumerate} \item  (link) \href{Access Continuum (Dictionary Entry) }{ }   \item  (link) \href{Access Control List (Dictionary Entry) }{ } \item  (link) \href{Access Granularity (Dictionary Entry) }{ }   \item  (link) \href{Entity (Dictionary Entry) }{ }   \item  Entitlement  \item  (link) \href{Identity (Dictionary Entry) }{ }   \item  (link) \href{Object (Dictionary Entry) }{ }   \item  (link) \href{Principal (Dictionary Entry) }{ }   \item  Privilege  \item  (link) \href{Stability of Access Decision Factors (Dictionary Entry) }{ }   \item  (link) \href{Subject (Dictionary Entry) }{ }   \end{enumerate} \subsection*{Quotes } \begin{DIC_BlockQuote} AccessTo make contact with one or more discrete functions of an online, digital service.  \end{DIC_BlockQuote} (NIST SP 800-63-3-R3, 2020, p. 39)  \paragraph{} (  (link) \href{NIST SP 800-63-3-R3, 2020 }{ } , p. 39)  \begin{DIC_BlockQuote} \$ access(I) The ability and means to communicate with or otherwise interact with a system in order to use system resources to either handle information or gain knowledge of the information the system contains.(O) "A specific type of interaction between a subject and an object that results in the flow of information from one to the other." {[}NCS04{]}(C) In this Glossary, "access" is intended to cover any ability to communicate with a system, including one-way communication in either direction. In actual practice, however, entities outside a security perimeter that can receive output from the system but cannot provide input or otherwise directly interact with the system, might be treated as not having "access" and, therefore, be exempt from security policy requirements, such as the need for a security clearance.  \end{DIC_BlockQuote} (RfC 2828, 2000, p. 7)  \paragraph{} (  (link) \href{RfC 2828, 2000 }{ } , p. 7)  \begin{DIC_BlockQuote} AccessThe ability to make use of information stored in a computer system. Used frequently as a verb, to the horror of grammarians.  \end{DIC_BlockQuote} (Saltzer and Schroeder, 1975)  \paragraph{} (  (link) \href{Saltzer and Schroeder, 1975 }{ } )  \subsection*{Bibliography } \begin{enumerate} \item  (link) \href{NIST SP 800-63-3-R3, 2020 }{ } \item  (link) \href{RfC 2828, 2000 }{ } \item  (link) \href{Saltzer and Schroeder, 1975 }{ } \end{enumerate} \subsection*{See Also } 