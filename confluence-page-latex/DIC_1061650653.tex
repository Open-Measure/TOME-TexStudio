\newpage \subsection*{Orphan Object } \subsection*{Alternative Forms } \begin{enumerate} \item  Orphan  \end{enumerate} \subsection*{Definitions } \begin{DIC_Def}{Definition 1Computer Science, Memory Management }{} \paragraph{} An object instance that is no longer referenced.  \end{DIC_Def} \subsection*{Related Terms } \begin{enumerate} \item  Object  \item  (link) \href{Orphan (Dictionary Entry) }{ }   \end{enumerate} \subsection*{Quotes } \begin{DIC_BlockQuote} Orphaned objects. The ability to assign different objects to a reference variable creates the possibility that a program may have created an object that it can no longer reference. For example, consider the three assignment statements in the figure at right. After the third assignment statement, not only do a and b refer to the same Color object (the one whose RGB values are 160, 82, and 45), but also there is no longer a reference to the Color object that was created and used to initialize b. The only reference to that object was in the variable b, and this reference was overwritten by the assignment, so there is no way to refer to the object again. Such an object is said to be orphaned. Objects are also orphaned when they go out of scope. Java programmers pay little attention to orphaned objects because the system automatically reuses the memory that they occupy, as we discuss next.  \end{DIC_BlockQuote} (Sedgewick and Wayne, 2017, p. 366)  \paragraph{} (  (link) \href{Sedgewick and Wayne, 2017 }{ } , p. 366)  \subsection*{Bibliography } \begin{enumerate} \item  (link) \href{Sedgewick and Wayne, 2017 }{ }   \end{enumerate} \subsection*{See Also } 