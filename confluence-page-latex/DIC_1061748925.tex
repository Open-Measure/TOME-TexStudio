\newpage \subsection*{Orphan Process } \subsection*{Alternative Forms } \begin{enumerate} \item  Orphan  \end{enumerate} \subsection*{Definitions } \begin{DIC_Def}{Definition 1Computer Science, Linux }{} \paragraph{} In Linux, a process whose parent is dead. An orphan process is automatically made a child of init.  \end{DIC_Def} \subsection*{Related Terms } \begin{enumerate} \item  Linux  \item  (link) \href{Orphan (Dictionary Entry) }{ }   \item  Process  \item  UNIX  \end{enumerate} \subsection*{Quotes } \begin{DIC_BlockQuote} Before a process can be allowed to disappear completely, the kernel requires thatits death be acknowledged by the process's parent, which the parent does with acall to wait. The parent receives a copy of the child's exit code (or an indication ofwhy the child was killed if the child did not exit voluntarily) and can also obtain asummary of the child's use of resources if it wishes.This scheme works fine if parents outlive their children and are conscientiousabout calling wait so that dead processes can be disposed of. If the parent diesfirst, however, the kernel recognizes that no wait will be forthcoming and adjuststhe process to make the orphan a child of init. init politely accepts these orphanedprocesses and performs the wait needed to get rid of them when they die.  \end{DIC_BlockQuote} (Nemeth et al., 2011, p. 124)  \paragraph{} (  (link) \href{Nemeth et al., 2011 }{ } , p. 124)  \subsection*{Bibliography } \begin{enumerate} \item  (link) \href{Nemeth et al., 2011 }{ }   \end{enumerate} \subsection*{See Also } label = "orphan-process"  false  title  