\newpage \subsection*{Account } \subsection*{Alternative Forms } \begin{enumerate} \item  a/c  \texttt{ Abbreviation  } \item  Acc.  \texttt{ Abbreviation  } \item  Digital Account  \texttt{ more precise form  } \item  Identity  \texttt{ Near Synonym  } \item  User Account  \texttt{ Synonym  }   \end{enumerate} \subsection*{Definitions } \begin{DIC_Def}{Definition 1 }{} \paragraph{} In the literature,  \emph{ account  } is generally used a synonym for  \emph{ identity  } .  \paragraph{} Because the definition of  \emph{ identity  } is defined as a set of identifier, credential and other attributes linked to an  \emph{ entity  } ,  \emph{ account  } could be used to distinguish technical and privileged accounts that are not necessarily linked to an entity, from  \emph{ identities  } or  \emph{ user accounts  } that are linked to an  \emph{ entity  } . But even though this nuance would be meaningful, the fact the both terms have been used interchangeably urges caution as the reader will most probably not understand this nuance.  \paragraph{} We recommend the usage of  \emph{ identity  } and the abandonment of  \emph{ account  } except when account is used in a fixed expression such as  \emph{ Windows account  } .  \end{DIC_Def} \subsection*{Related Terms } \begin{enumerate} \item  (link) \href{Identity (Dictionary Entry) }{ } \item  (link) \href{Orphan Account (Dictionary Entry) }{ } \item  User Account  \end{enumerate} \subsection*{Quotes } \begin{DIC_BlockQuote} account The environment in which a user interacts with a computer system. Each account has a unique name, which the user specifies when logging in. System data associated with an account controls what resources (files, programs, networks, etc.) the user can access and in what ways (e.g. whether files that are normally writeable are read-only for particular users) and to what extent (e.g. the total size of files a user creates may be limited). Where applicable, a record can be kept of the resources used for billing purposes.  \end{DIC_BlockQuote} (Butterfield et al., 2016, p. 109)  \paragraph{} (  (link) \href{Butterfield et al., 2016 }{ } , p. 109)  \begin{DIC_BlockQuote} account authorization to use a computer or any kind of computer service, even if free of charge. An account consists of an identifying name and other records necessary to keep track of a user. Sometimes an account belongs to another computer or a computer program rather than a human being.  \end{DIC_BlockQuote} (Downing et al., 2009, p. 10)  \paragraph{} (  (link) \href{Downing et al., 2009 }{ } , p. 10)  \begin{DIC_BlockQuote} An entity's access to a system is encapsulated in what has become known as an account.  \end{DIC_BlockQuote} (Benantar, 2006, p. 3)  \paragraph{} (  (link) \href{Benantar, 2006 }{ } , p. 3)  \begin{DIC_BlockQuote} The term user in computing has been traditionally equated with a human being. Its use conveys a unique association between a computing system and an entity that can be a human being or some programmable agent. User information is generally encapsulated in an account, sometimes referred to as a profile. A user account contains information about authentication as well as authorization credentials and may contain a set of attributes describing the user (such as a name, a serial number, an organization name, and so forth). Each user account is associated with an identifier that must be unique in the naming space of the underlying computing system.  \end{DIC_BlockQuote} (Benantar, 2006, p. 9)  \paragraph{} (  (link) \href{Benantar, 2006 }{ } , p. 9)  \subsection*{Bibliography } \begin{enumerate} \item  (link) \href{Benantar, 2006 }{ }   \item  (link) \href{Butterfield et al., 2016 }{ }   \item  (link) \href{Downing et al., 2009 }{ }   \end{enumerate} \subsection*{See Also } 