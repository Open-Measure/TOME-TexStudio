\newpage \subsection*{AWS IAM User } \subsection*{Definitions } \begin{DIC_Def}{Definition 1AWS }{} \paragraph{} An identity in AWS. It is mapped to either a person or an application. It has 3 identifiers: a friendly name, an ARN and a unique ID. It is linked to a single  \emph{ AWS Account  } . It may be a member of  \emph{ AWS IAM Groups  } . It may be granted direct permissions or indirect permissions via  \emph{ AWS IAM Group  } membership.  \paragraph{} The  \emph{ AWS Account Root User  } is not considered as an  \emph{ AWS IAM User  } .  \end{DIC_Def} \subsection*{Related Terms } \begin{enumerate} \item  ARN  \item  AWS  \item  AWS Account  \item  AWS Account Root User (Dictionary Entry)    \item  AWS IAM  \item  AWS IAM Group (Dictionary Entry)    \end{enumerate} \subsection*{Quotes } \begin{DIC_BlockQuote} IAM UserAn AWS Identity and Access Management (IAM)~user~is an entity that you create in AWS to represent the person or application that uses it to interact with AWS. A user in AWS consists of a name and credentials.  \end{DIC_BlockQuote} (AWS, 11/2020, p. 74)\\ (Online: \url{https://docs.aws.amazon.com/IAM/latest/UserGuide/id_users.html})  \paragraph{} (  AWS, 11/2020  , p. 74)  (Online:  https://docs.aws.amazon.com/IAM/latest/UserGuide/id\_users.html  )  \subsection*{Bibliography } \begin{enumerate} \item  AWS, 11/2020    \end{enumerate} \subsection*{See Also } false  title  label = "aws-iam-user"  