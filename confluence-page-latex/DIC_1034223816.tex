\newpage \subsection*{Access Continuum } \subsection*{Definitions } \begin{DIC_Def}{Definition 1 }{} \paragraph{} Entities have varying levels of access to organizations' resources. The binary classification  \emph{ insider  } versus  \emph{ outsider  } is a highly simplified model. In contrast, considering access levels as a continuum allows for a more sophisticated model and may help focus on the most critical aspect: access, rather than statute.  \end{DIC_Def} \subsection*{Illustration } \subsection*{Related Terms } \begin{enumerate} \item  (link) \href{Access (Dictionary Entry) }{ }   \item  Continuum  \item  Insider  \item  Insider Threat  \item  Level of Access  \item  Outsider  \item  Outsider Threat  \end{enumerate} \subsection*{Quotes } \begin{DIC_BlockQuote} Our theme is that the distinction between ``insider'' and ``outsider'' is not binary; rather, there are ``attackers'' with varying degrees and types of access. One can call some set of these attackers ``insiders,'' with the complement being the ``outsiders,'' but countermeasures should focus on the access and not on whether the attackers are insiders. Thus, we see attacks as spanning a continuum of levels and types of access, and use that as the basis of our discussion. We emphasize that people comfortable thinking in terms of ``insiders'' and ``outsiders'' can superimpose that partition on our notion of ``attackers with varying levels of access.'' That partition, however, will vary based on circumstances and environment.  \end{DIC_BlockQuote} (Bishop et al., 2010, p. 117)  \paragraph{} (  (link) \href{Bishop et al., 2010 }{ } , p. 117)  \subsection*{Bibliography } \begin{enumerate} \item  (link) \href{Bishop et al., 2010 }{ }   \end{enumerate} \subsection*{See Also } 