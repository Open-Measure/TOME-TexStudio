\newpage \subsection*{IAM ARN } \subsection*{Definitions } \begin{DIC_Def}{Definition 1AWS }{} \paragraph{} The ARN of an AWS IAM object, in the form:  \texttt{ arn:partition:service:region:account:resource  } .  \end{DIC_Def} \subsection*{Related Terms } \begin{enumerate} \item  ARN  \item  AWS  \item  URN  \end{enumerate} \subsection*{Quotes } \begin{DIC_BlockQuote} IAM ARNsMost resources have a friendly name (for example, a user named Bob or a group named Developers). However, the permissions policy language requires you to specify the resource or resources using the following Amazon Resource Name (ARN) format.arn:partition:service:region:account:resource  \end{DIC_BlockQuote} (AWS, 11/2020, p. 597)\\ (Online: \url{https://docs.aws.amazon.com/IAM/latest/UserGuide/reference_identifiers.html\#identifiers-arns})  \paragraph{} (  (link) \href{AWS, 11/2020 }{ } , p. 597)  \\  (Online:  \href{None }{https://docs.aws.amazon.com/IAM/latest/UserGuide/reference\_identifiers.html\#identifiers-arns } )  \subsection*{Bibliography } \begin{enumerate} \item  (link) \href{AWS, 11/2020 }{ }   \end{enumerate} \subsection*{See Also } 