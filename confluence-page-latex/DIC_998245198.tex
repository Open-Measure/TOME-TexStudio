\newpage \subsection*{AWS ACL } \subsection*{Alternate Forms } \begin{enumerate} \item  AWS Access Control List  \end{enumerate} \subsection*{Definitions } \begin{DIC_Def}{Definition 1AWS }{} \paragraph{} An ACL implementation specific to AWS whose scope is limited to granting access to identities outside the AWS Account that contains the resource. Contrary to other AWS policy types, AWS ACL is not following the AWS JSON policy format.  \end{DIC_Def} \subsection*{Related Terms } \begin{enumerate} \item  (link) \href{Access Control List (Dictionary Entry) }{ }   \texttt{ Generic Form  } \item  AWS  \item  AWS Account  \item  AWS IAM  \item  (link) \href{AWS IAM Policy (Dictionary Entry) }{ }   \end{enumerate} \subsection*{Quotes } \begin{DIC_BlockQuote} Access control lists (ACLs)Access control lists (ACLs) are service policies that allow you to control which principals in another account can access a resource. ACLs cannot be used to control access for a principal within the same account. ACLs are similar to resource-based policies, although they are the only policy type that does not use the JSON policy document format. Amazon S3, AWS WAF, and Amazon VPC are examples of services that support ACLs.  \end{DIC_BlockQuote} (AWS, 11/2020, p. 353)\\ (Online: \url{https://docs.aws.amazon.com/IAM/latest/UserGuide/id_groups.html})  \paragraph{} (  (link) \href{AWS, 11/2020 }{ } , p. 353)  (Online:  https://docs.aws.amazon.com/IAM/latest/UserGuide/id\_groups.html  )  \subsection*{Bibliography } \begin{enumerate} \item  (link) \href{AWS, 11/2020 }{ }   \end{enumerate} \subsection*{See Also } false  title  label in ( "aws-access-control-list" , "aws-acl" )  