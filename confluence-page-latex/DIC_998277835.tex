\newpage \subsection*{AWS IAM Temporary Security Credentials } \subsection*{Definitions } \begin{DIC_Def}{Definition 1AWS }{} \paragraph{} A temporary identity in AWS.  \end{DIC_Def} \subsection*{Related Terms } \begin{enumerate} \item  AWS  \item  AWS Account  \item  AWS Account Root User (Dictionary Entry)    \item  AWS IAM  \item  AWS IAM Group (Dictionary Entry)    \item  AWS IAM Role (Dictionary Entry)    \item  AWS IAM User (Dictionary Entry)    \end{enumerate} \subsection*{Quotes } \begin{DIC_BlockQuote} Temporary security credentials in IAMYou can use the AWS Security Token Service (AWS STS) to create and provide trusted users with temporary security credentials that can control access to your AWS resources. Temporary security credentials work almost identically to the long-term access key credentials that your IAM users can use, with the following differences: • Temporary security credentials are~short-term, as the name implies. They can be configured to last for anywhere from a few minutes to several hours. After the credentials expire, AWS no longer recognizes them or allows any kind of access from API requests made with them. • Temporary security credentials are not stored with the user but are generated dynamically and provided to the user when requested. When (or even before) the temporary security credentials expire, the user can request new credentials, as long as the user requesting them still has permissions to do so.These differences lead to the following advantages for using temporary credentials: • You do not have to distribute or embed long-term AWS security credentials with an application. • You can provide access to your AWS resources to users without having to define an AWS identity for them. Temporary credentials are the basis for~roles and identity federation. • The temporary security credentials have a limited lifetime, so you do not have to rotate them or explicitly revoke them when they're no longer needed. After temporary security credentials expire, they cannot be reused. You can specify how long the credentials are valid, up to a maximum limit.  \end{DIC_BlockQuote} (AWS, 11/2020, p. 301)\\ (Online: \url{https://docs.aws.amazon.com/IAM/latest/UserGuide/id_credentials_temp.html})  \paragraph{} (  AWS, 11/2020  , p. 301)  (Online:  https://docs.aws.amazon.com/IAM/latest/UserGuide/id\_credentials\_temp.html  )  \subsection*{Bibliography } \begin{enumerate} \item  AWS, 11/2020    \end{enumerate} \subsection*{See Also } label = "aws-iam-temporary-security-credentials"  false  title  