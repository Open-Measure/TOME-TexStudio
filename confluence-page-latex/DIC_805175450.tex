\newpage \subsection*{4-Eyes Principle } \subsection*{Alternative Forms } \begin{enumerate} \item  2-Person Control  \item  4-Eyes Check  \item  Dual Authorization  \item  Four-Eyes Check  \item  Four-Eyes Principle  \item  Two-Person Control  \end{enumerate} \subsection*{Definitions } \begin{DIC_Def}{Definition 1AuditIAMAccess Management }{} \paragraph{} A type of segregation of duties control which prescribes that crucial decisions or operations be prepared by one actor, and reviewed and validated by another actor before the decision or the operation becomes effective. This setup contrasts with decisions or operations that may be made by individual actors. This control mitigates the risk of inconsiderate, fraudulent or suboptimal decisions.  \end{DIC_Def} \subsection*{Related Terms } \begin{enumerate} \item    (link) \href{Dual Authorization (Dictionary Entry) }{ }   \end{enumerate} \subsection*{Quotes } \begin{DIC_BlockQuote} An instrument often used to reduce the risk of inconsiderate, fraudulent or suboptimal decisions in all types of organizations -- and not just in family firms -- is the four-eyes principle (4EP) (Sutter, 2007; Feldbauer-Durstmüller et al., 2012; Six et al., 2012; Bátiz-Lazo and Noguchi, 2013){[}1{]}. This principle usually means that crucial decisions (often defined as those affecting a certain minimum amount of capital) may not be made by individual actors alone but must be jointly made by at least two actors. The inclusion of at least two actors also explains why the principle's name includes ``four eyes''. This approach ensures the rationality of decisions as well as reciprocal control of decisions (Schickora, 2010). This is why Gottschalk (2011, p. 300), based on a study of executives involved in white-collar crime, states that ``the 4EP should always be applied'' in management decisions. In addition, the 4EP not only mitigates high-risk decision outcomes but also enriches the decision-making process by integrating the views of different individuals (Knoll, 2013).  \end{DIC_BlockQuote} (Hiebl, 2015, p. 1-2)  \paragraph{} (  (link) \href{Hiebl, 2015 }{ } , p. 1-2)  \subsection*{Bibliography } \begin{enumerate} \item  (link) \href{Hiebl, 2015 }{ }   \end{enumerate} \subsection*{See Also } 