% DOCUMENT CLASS

\documentclass[11pt,fleqn]{book} % Default font size and left-justified equations

% UNICODE CHARACTERS

% \usepackage[utf8]{inputenc}
% \usepackage{unixode}

% COLORS

\usepackage{xcolor} % Required for specifying colors by name
\definecolor{White_Color}{RGB}{255,255,255} 
\definecolor{DarkBlue_Color}{RGB}{0,102,255} 
\definecolor{StandardBlue_Color}{RGB}{85,153,255} 
\definecolor{LightBlue_Color}{RGB}{170,204,255} 
\definecolor{DarkGrey_Color}{RGB}{153,153,153} 
\definecolor{StandardGrey_Color}{RGB}{179,179,179} 





%----------------------------------------------------------------------------------------
%	HEADERS AND FOOTERS
%----------------------------------------------------------------------------------------

\usepackage{fancyhdr} % Required for header and footer configuration

\pagestyle{fancy} % Enable the custom headers and footers

\renewcommand{\chaptermark}[1]{\markboth{\sffamily\normalsize\bfseries\chaptername\ \thechapter.\ #1}{}} % Styling for the current chapter in the header
\renewcommand{\sectionmark}[1]{\markright{\sffamily\normalsize\thesection\hspace{5pt}#1}{}} % Styling for the current section in the header

\fancyhf{} % Clear default headers and footers
\fancyhead[LE,RO]{\sffamily\normalsize\thepage} % Styling for the page number in the header
\fancyhead[LO]{\rightmark} % Print the nearest section name on the left side of odd pages
\fancyhead[RE]{\leftmark} % Print the current chapter name on the right side of even pages
%\fancyfoot[C]{\thepage} % Uncomment to include a footer

\renewcommand{\headrulewidth}{0.5pt} % Thickness of the rule under the header

\fancypagestyle{plain}{% Style for when a plain pagestyle is specified
	\fancyhead{}\renewcommand{\headrulewidth}{0pt}%
}

% Removes the header from odd empty pages at the end of chapters
\makeatletter
\renewcommand{\cleardoublepage}{
	\clearpage\ifodd\c@page\else
	\hbox{}
	\vspace*{\fill}
	\thispagestyle{empty}
	\newpage
	\fi}


%	PART HEADINGS


%	CHAPTER HEADINGS


% HYPERLINKS

\usepackage{hyperref}
\hypersetup{
	hidelinks,
	backref=true,
	pagebackref=true,
	hyperindex=true,
	colorlinks=false,
	breaklinks=true,
	linkcolor=DarkBlue_Color,
	filecolor=DarkBlue_Color,      
	urlcolor=DarkBlue_Color,
	bookmarks=true,
	bookmarksopen=false
}

%\urlstyle{same}

\usepackage{bookmark}
\bookmarksetup{
	open,
	numbered,
	addtohook={%
		\ifnum\bookmarkget{level}=0 % chapter
		\bookmarksetup{bold}%
		\fi
		\ifnum\bookmarkget{level}=-1 % part
		\bookmarksetup{color=StandardBlue_Color,bold}%
		\fi
	}
}


% THEOREMS

\usepackage{amsthm}

% SECTION HEADERS FORMATTING

\usepackage{titlesec}

\titleformat{\subsubsection}
{\normalfont\fontsize{12}{15}\bfseries}{\thesection}{1em}{}

% BOXES

\usepackage[many]{tcolorbox}
\tcbuselibrary{theorems}

% DICTIONARY DEFINITIONS

\newtcbtheorem[]{DIC_Def}{Definition}%
{colback=White_Color,colframe=StandardBlue_Color,fonttitle=\bfseries}{th}

% LISTS

\usepackage{enumitem} % Customize lists
% \setlist{parsep=0pt,listparindent=\parindent}
% \begin{enumerate}
\setlist{nolistsep} % Reduce spacing between bullet points and numbered lists


% BLOCK QUOTES

% References:
% - https://tex.stackexchange.com/questions/277777/block-quote-with-big-quotation-marks-in-margin

\newtcolorbox{DIC_BlockQuote}{%
	enhanced jigsaw, 
	breakable,      % allow page breaks
	frame hidden,   % hide the default frame
	left=0cm,       % left margin
	right=0cm,      % right margin
	overlay={%
		\node [scale=6,
		text=StandardBlue_Color,
		inner sep=0pt,] at ([xshift=-0.6cm,yshift=-1cm]frame.north west){``}; 
		\node [scale=6,
		text=StandardBlue_Color,
		inner sep=0pt,] at ([xshift=0.6cm]frame.south east){''};  
	},
	% paragraph skips obeyed within tcolorbox
	parbox=false,
}

% PICTURES

\usepackage{graphicx} % Required for including pictures
\graphicspath{{Pictures/}} % Specifies the directory where pictures are stored

% MISC.

\usepackage{tikz} % Required for drawing custom shapes
\usepackage{booktabs} % Required for nicer horizontal rules in tables

%	MARGINS

\usepackage{geometry} % Required for adjusting page dimensions and margins
\geometry{
	paper=a4paper, % Paper size, change to letterpaper for US letter size
	top=3cm, % Top margin
	bottom=3cm, % Bottom margin
	left=3cm, % Left margin
	right=3cm, % Right margin
	headheight=14pt, % Header height
	footskip=1.4cm, % Space from the bottom margin to the baseline of the footer
	headsep=10pt, % Space from the top margin to the baseline of the header
	%showframe, % Uncomment to show how the type block is set on the page
}

% FONTS

\usepackage{avant} % Use the Avantgarde font for headings
%\usepackage{times} % Use the Times font for headings
\usepackage{mathptmx} % Use the Adobe Times Roman as the default text font together with math symbols from the Sym­bol, Chancery and Com­puter Modern fonts

\usepackage{microtype} % Slightly tweak font spacing for aesthetics
\usepackage[utf8]{inputenc} % Required for including letters with accents
\usepackage[T1]{fontenc} % Use 8-bit encoding that has 256 glyphs


%	SECTION NUMBERING IN THE MARGIN

\makeatletter
\renewcommand{\@seccntformat}[1]{\llap{\textcolor{StandardBlue_Color}{\csname the#1\endcsname}\hspace{1em}}}                    
\renewcommand{\section}{\@startsection{section}{1}{\z@}
	{-4ex \@plus -1ex \@minus -.4ex}
	{1ex \@plus.2ex }
	{\normalfont\large\sffamily\bfseries}}
\renewcommand{\subsection}{\@startsection {subsection}{2}{\z@}
	{-3ex \@plus -0.1ex \@minus -.4ex}
	{0.5ex \@plus.2ex }
	{\normalfont\sffamily\bfseries}}
\renewcommand{\subsubsection}{\@startsection {subsubsection}{3}{\z@}
	{-2ex \@plus -0.1ex \@minus -.2ex}
	{.2ex \@plus.2ex }
	{\normalfont\small\sffamily\bfseries}}                        
\renewcommand\paragraph{\@startsection{paragraph}{4}{\z@}
	{-2ex \@plus-.2ex \@minus .2ex}
	{.1ex}
	{\normalfont\small\sffamily\bfseries}}


% DOCUMENT

\begin{document}
	
%	TITLE PAGE

\begingroup
\thispagestyle{empty} % Suppress headers and footers on the title page
\begin{tikzpicture}[remember picture,overlay]
%%	\node[inner sep=0pt] (background) at (current page.center) {\includegraphics[width=\paperwidth]{background.pdf}};
	\draw (current page.center) node [fill=LightBlue_Color!30!white,fill opacity=0.6,text opacity=1,inner sep=1cm]{\Huge\centering\bfseries\sffamily\parbox[c][][t]{\paperwidth}{\centering The Open-Measure Encyclopedia\\[15pt] % Book title
			{\Large Identity and Access Management}\\[20pt] % Subtitle
			{\huge David DORET}}}; % Author name
\end{tikzpicture}
\vfill
\endgroup

%	COPYRIGHT PAGE

\newpage
~\vfill
\thispagestyle{empty}

\noindent Copyright \copyright\ 2020 Open-Measure\\ % Copyright notice

\noindent \textsc{Published by Open-Measure}\\ % Publisher

\noindent \textsc{https://open-measure.org}\\ % URL

\noindent Licensed under the Creative Commons Attribution-NonCommercial 3.0 Unported License (the ``License''). You may not use this file except in compliance with the License. You may obtain a copy of the License at \url{http://creativecommons.org/licenses/by-nc/3.0}. Unless required by applicable law or agreed to in writing, software distributed under the License is distributed on an \textsc{``as is'' basis, without warranties or conditions of any kind}, either express or implied. See the License for the specific language governing permissions and limitations under the License.\\ % License information, replace this with your own license (if any)

\noindent \textit{First printing, December 2020} % Printing/edition date

%	TABLE OF CONTENTS

%\usechapterimagefalse % If you don't want to include a chapter image, use this to toggle images off - it can be enabled later with \usechapterimagetrue

%% \chapterimage{chapter_head_1.pdf} % Table of contents heading image

\pagestyle{empty} % Disable headers and footers for the following pages

\tableofcontents % Print the table of contents itself

\cleardoublepage % Forces the first chapter to start on an odd page so it's on the right side of the book

%% \pagestyle{fancy} % Enable headers and footers again

%----------------------------------------------------------------------------------------
%	PART
%----------------------------------------------------------------------------------------

\part{Part 1}		
		
	\paragraph{} content \emph{text}
	
	\subsubsection*{Term}{
		\begin{DIC_Def}{}{}This is a definition.\end{DIC_Def}
		\begin{DIC_Def}{Context}{}This is a definition.\end{DIC_Def}
		\begin{DIC_BlockQuote}This is a quote.\end{DIC_BlockQuote}
	}

	\subsubsection*{Another Term}{
	\begin{DIC_Def}{}{}This is a definition.\end{DIC_Def}
	\begin{DIC_Def}{Context}{}This is a definition.\end{DIC_Def}
	\begin{DIC_BlockQuote}This is a quote.\end{DIC_BlockQuote}
	}

\part{Part 2}	


% Generation time: 2021-01-08T23:18:17.584366 
 % Title: DICTIONARY TERMS INCLUSION LIST 
 % Description: This file references all the dictionary term files with an \input command 
 
 % Page Space: DIC 
 % Page ID: 998277876 
 % Page Title: AWS IAM (Dictionary Entry) 
 \newpage \subsection*{AWS IAM } \subsection*{Definitions } \begin{DIC_Def}{Definition 1AWS }{} \paragraph{} The native IAM platform in AWS.  \end{DIC_Def} \subsection*{Related Terms } \begin{enumerate} \item  AWS  \item  AWS Account  \item  AWS Account Root User (Dictionary Entry)    \item  AWS IAM Group (Dictionary Entry)    \item  AWS IAM Policy (Dictionary Entry)    \item  AWS IAM Role (Dictionary Entry)    \item  AWS IAM Temporary Security Credentials (Dictionary Entry)    \item  AWS IAM User (Dictionary Entry)    \end{enumerate} \subsection*{Quotes } \begin{DIC_BlockQuote} AWS Identity and Access Management (IAM) is a web service that helps you securely control access to AWS resources. You use IAM to control who is authenticated (signed in) and authorized (has permissions) to use resources.  \end{DIC_BlockQuote} (AWS, 11/2020, p. 1)\\ (Online: \url{https://docs.aws.amazon.com/IAM/latest/UserGuide/introduction.html})  \paragraph{} (  AWS, 11/2020  , p. 1)  (Online:  https://docs.aws.amazon.com/IAM/latest/UserGuide/introduction.html  )  \subsection*{Bibliography } \begin{enumerate} \item  AWS, 11/2020    \end{enumerate} \subsection*{See Also } label = "aws-iam"  false  title   
 
 % Page Space: DIC 
 % Page ID: 998277852 
 % Page Title: AWS IAM Policy (Dictionary Entry) 
 \newpage \subsection*{AWS IAM Policy } \subsection*{Definitions } \begin{DIC_Def}{Definition 1AWS }{} \paragraph{} An access policy in AWS.  \end{DIC_Def} \subsection*{Related Terms } \begin{enumerate} \item  AWS  \item  AWS Account  \item  AWS Account Root User (Dictionary Entry)    \item  AWS IAM  \item  AWS IAM Group (Dictionary Entry)    \item  AWS IAM Role (Dictionary Entry)    \item  AWS IAM User (Dictionary Entry)    \end{enumerate} \subsection*{Quotes } \begin{DIC_BlockQuote} Policies and Permissions in IAMYou manage access in AWS by creating policies and attaching them to IAM identities (users, groups of users, or roles) or AWS resources. A policy is an object in AWS that, when associated with an identity or resource, defines their permissions. AWS evaluates these policies when an IAM principal (user or role) makes a request. Permissions in the policies determine whether the request is allowed or denied. Most policies are stored in AWS as JSON documents. AWS supports six types of policies: identity-based policies, resource-based policies, permissions boundaries, Organizations SCPs, ACLs, and session policies.IAM policies define permissions for an action regardless of the method that you use to perform the operation. For example, if a policy allows the~GetUser~action, then a user with that policy can get user information from the AWS Management Console, the AWS CLI, or the AWS API. When you create an IAM user, you can choose to allow console or programmatic access. If console access is allowed, the IAM user can sign in to the console using a user name and password. Or if programmatic access is allowed, the user can use access keys to work with the CLI or API..  \end{DIC_BlockQuote} (AWS, 11/2020, p. 351)\\ (Online: \url{https://docs.aws.amazon.com/IAM/latest/UserGuide/access_policies.html})  \paragraph{} (  AWS, 11/2020  , p. 351)  (Online:  https://docs.aws.amazon.com/IAM/latest/UserGuide/access\_policies.html  )  \subsection*{Bibliography } \begin{enumerate} \item  AWS, 11/2020    \end{enumerate} \subsection*{See Also } label = "aws-iam-policy"  false  title   
 
 % Page Space: DIC 
 % Page ID: 998277835 
 % Page Title: AWS IAM Temporary Security Credentials (Dictionary Entry) 
 \newpage \subsection*{AWS IAM Temporary Security Credentials } \subsection*{Definitions } \begin{DIC_Def}{Definition 1AWS }{} \paragraph{} A temporary identity in AWS.  \end{DIC_Def} \subsection*{Related Terms } \begin{enumerate} \item  AWS  \item  AWS Account  \item  (link) \href{AWS Account Root User (Dictionary Entry) }{ }   \item  AWS IAM  \item  (link) \href{AWS IAM Group (Dictionary Entry) }{ }   \item  (link) \href{AWS IAM Role (Dictionary Entry) }{ }   \item  (link) \href{AWS IAM User (Dictionary Entry) }{ }   \end{enumerate} \subsection*{Quotes } \begin{DIC_BlockQuote} Temporary security credentials in IAMYou can use the AWS Security Token Service (AWS STS) to create and provide trusted users with temporary security credentials that can control access to your AWS resources. Temporary security credentials work almost identically to the long-term access key credentials that your IAM users can use, with the following differences: • Temporary security credentials are~short-term, as the name implies. They can be configured to last for anywhere from a few minutes to several hours. After the credentials expire, AWS no longer recognizes them or allows any kind of access from API requests made with them. • Temporary security credentials are not stored with the user but are generated dynamically and provided to the user when requested. When (or even before) the temporary security credentials expire, the user can request new credentials, as long as the user requesting them still has permissions to do so.These differences lead to the following advantages for using temporary credentials: • You do not have to distribute or embed long-term AWS security credentials with an application. • You can provide access to your AWS resources to users without having to define an AWS identity for them. Temporary credentials are the basis for~roles and identity federation. • The temporary security credentials have a limited lifetime, so you do not have to rotate them or explicitly revoke them when they're no longer needed. After temporary security credentials expire, they cannot be reused. You can specify how long the credentials are valid, up to a maximum limit.  \end{DIC_BlockQuote} (AWS, 11/2020, p. 301)\\ (Online: \url{https://docs.aws.amazon.com/IAM/latest/UserGuide/id_credentials_temp.html})  \paragraph{} (  (link) \href{AWS, 11/2020 }{ } , p. 301)  (Online:  https://docs.aws.amazon.com/IAM/latest/UserGuide/id\_credentials\_temp.html  )  \subsection*{Bibliography } \begin{enumerate} \item  (link) \href{AWS, 11/2020 }{ }   \end{enumerate} \subsection*{See Also } label = "aws-iam-temporary-security-credentials"  false  title   
 
 % Page Space: DIC 
 % Page ID: 998277805 
 % Page Title: AWS IAM Group (Dictionary Entry) 
 \newpage \subsection*{AWS IAM Group } \subsection*{Definitions } \begin{DIC_Def}{Definition 1AWS }{} \paragraph{} A security group in AWS. It contains  \emph{ AWS IAM Users  } and may be granted permissions via policies. It has a flat structure, i.e. AWS IAM does not support group nesting.  \end{DIC_Def} \subsection*{Related Terms } \begin{enumerate} \item  ARN  \item  AWS  \item  AWS Account  \item  AWS IAM  \item  (link) \href{AWS IAM User (Dictionary Entry) }{ }   \item  Group  \end{enumerate} \subsection*{Quotes } \begin{DIC_BlockQuote} IAM GroupsAn IAM group is a collection of IAM users. Groups let you specify permissions for multiple users, which can make it easier to manage the permissions for those users. For example, you could have a group called Admins and give that group the types of permissions that administrators typically need. Any user in that group automatically has the permissions that are assigned to the group. If a new user joins your organization and needs administrator privileges, you can assign the appropriate permissions by adding the user to that group. Similarly, if a person changes jobs in your organization, instead of editing that user's permissions, you can remove him or her from the old groups and add him or her to the appropriate new groups.Note that a group is not truly an "identity" in IAM because it cannot be identified as a Principal in a permission policy. It is simply a way to attach policies to multiple users at one time.Following are some important characteristics of groups:- A group can contain many users, and a user can belong to multiple groups.- Groups can't be nested; they can contain only users, not other groups.- There's no default group that automatically includes all users in the AWS account. If you want to have a group like that, you need to create it and assign each new user to it.- The number and size of IAM resources in an AWS account are limited. For more information, see IAM and STS quotas.  \end{DIC_BlockQuote} (AWS, 11/2020, p. 160)\\ (Online: \url{https://docs.aws.amazon.com/IAM/latest/UserGuide/id_groups.html})  \paragraph{} (  (link) \href{AWS, 11/2020 }{ } , p. 160)  (Online:  https://docs.aws.amazon.com/IAM/latest/UserGuide/id\_groups.html  )  \subsection*{Bibliography } \begin{enumerate} \item  (link) \href{AWS, 11/2020 }{ }   \end{enumerate} \subsection*{See Also } false  title  label = "aws-iam-group"   
 
 % Page Space: DIC 
 % Page ID: 998277774 
 % Page Title: AWS Account Root User (Dictionary Entry) 
 \newpage \subsection*{AWS Account Root User } \subsection*{Definitions } \begin{DIC_Def}{Definition 1AWS }{} \paragraph{} The root user of an AWS account, with unlimited privileges on the account and its resources.  \end{DIC_Def} \subsection*{Related Terms } \begin{enumerate} \item  AWS  \item  Root User  \end{enumerate} \subsection*{Quotes } \begin{DIC_BlockQuote} AWS account root userWhen you first create an Amazon Web Services (AWS) account, you begin with a single sign-in identity that has complete access to all AWS services and resources in the account. This identity is called the AWS account root user and is accessed by signing in with the email address and password that you used to create the account.  \end{DIC_BlockQuote} (AWS, 11/2020, p. 72)\\ (Online: \url{https://docs.aws.amazon.com/IAM/latest/UserGuide/id.html})  \paragraph{} (  (link) \href{AWS, 11/2020 }{ } , p. 72)  (Online:  https://docs.aws.amazon.com/IAM/latest/UserGuide/id.html  )  \subsection*{Bibliography } \begin{enumerate} \item  (link) \href{AWS, 11/2020 }{ }   \end{enumerate} \subsection*{See Also } label = "aws-account-root-user"  false  title   
 
 % Page Space: DIC 
 % Page ID: 998245198 
 % Page Title: AWS ACL (Dictionary Entry) 
 \newpage \subsection*{AWS ACL } \subsection*{Alternate Forms } \begin{enumerate} \item  AWS Access Control List  \end{enumerate} \subsection*{Definitions } \begin{DIC_Def}{Definition 1AWS }{} \paragraph{} An ACL implementation specific to AWS whose scope is limited to granting access to identities outside the AWS Account that contains the resource. Contrary to other AWS policy types, AWS ACL is not following the AWS JSON policy format.  \end{DIC_Def} \subsection*{Related Terms } \begin{enumerate} \item  (link) \href{Access Control List (Dictionary Entry) }{ }   \texttt{ Generic Form  } \item  AWS  \item  AWS Account  \item  AWS IAM  \item  (link) \href{AWS IAM Policy (Dictionary Entry) }{ }   \end{enumerate} \subsection*{Quotes } \begin{DIC_BlockQuote} Access control lists (ACLs)Access control lists (ACLs) are service policies that allow you to control which principals in another account can access a resource. ACLs cannot be used to control access for a principal within the same account. ACLs are similar to resource-based policies, although they are the only policy type that does not use the JSON policy document format. Amazon S3, AWS WAF, and Amazon VPC are examples of services that support ACLs.  \end{DIC_BlockQuote} (AWS, 11/2020, p. 353)\\ (Online: \url{https://docs.aws.amazon.com/IAM/latest/UserGuide/id_groups.html})  \paragraph{} (  (link) \href{AWS, 11/2020 }{ } , p. 353)  \\  (Online:  \href{None }{https://docs.aws.amazon.com/IAM/latest/UserGuide/id\_groups.html } )  \subsection*{Bibliography } \begin{enumerate} \item  (link) \href{AWS, 11/2020 }{ }   \end{enumerate} \subsection*{See Also }  
 
 % Page Space: DIC 
 % Page ID: 998245151 
 % Page Title: AWS IAM Role (Dictionary Entry) 
 \newpage \subsection*{AWS IAM Role } \subsection*{Definitions } \begin{DIC_Def}{Definition 1AWS }{} \paragraph{} A temporary on-demand business role in AWS. Once an identity is granted permission to assume a role, the identity may assume that role by demanding it. It then inherits all of the access permissions linked to it.  \end{DIC_Def} \subsection*{Related Terms } \begin{enumerate} \item  AWS  \item  AWS Account  \item  (link) \href{AWS Account Root User (Dictionary Entry) }{ }   \item  AWS IAM  \item  (link) \href{AWS IAM Group (Dictionary Entry) }{ }   \item  (link) \href{AWS IAM User (Dictionary Entry) }{ }   \end{enumerate} \subsection*{Quotes } \begin{DIC_BlockQuote} IAM RolesAn IAM role is an IAM identity that you can create in your account that has specific permissions. An IAM role is similar to an IAM user, in that it is an AWS identity with permission policies that determine what the identity can and cannot do in AWS. However, instead of being uniquely associated with one person, a role is intended to be assumable by anyone who needs it. Also, a role does not have standard longterm credentials such as a password or access keys associated with it. Instead, when you assume a role, it provides you with temporary security credentials for your role session.You can use roles to delegate access to users, applications, or services that don't normally have access to your AWS resources. For example, you might want to grant users in your AWS account access to resources they don't usually have, or grant users in one AWS account access to resources in another account. Or you might want to allow a mobile app to use AWS resources, but not want to embed AWS keys within the app (where they can be difficult to rotate and where users can potentially extract them). Sometimes you want to give AWS access to users who already have identities defined outside of AWS, such as in your corporate directory. Or, you might want to grant access to your account to third parties so that they can perform an audit on your resources.For these scenarios, you can delegate access to AWS resources using an IAM role.  \end{DIC_BlockQuote} (AWS, 11/2020, p. 167)\\ (Online: \url{https://docs.aws.amazon.com/IAM/latest/UserGuide/id_roles.html})  \paragraph{} (  (link) \href{AWS, 11/2020 }{ } , p. 167)  \\  (Online:  \href{None }{https://docs.aws.amazon.com/IAM/latest/UserGuide/id\_roles.html } )  \subsection*{Bibliography } \begin{enumerate} \item  (link) \href{AWS, 11/2020 }{ }   \end{enumerate} \subsection*{See Also }  
 
 % Page Space: DIC 
 % Page ID: 998245125 
 % Page Title: AWS IAM User (Dictionary Entry) 
 \newpage \subsection*{AWS IAM User } \subsection*{Definitions } \begin{DIC_Def}{Definition 1AWS }{} \paragraph{} An identity in AWS. It is mapped to either a person or an application. It has 3 identifiers: a friendly name, an ARN and a unique ID. It is linked to a single  \emph{ AWS Account  } . It may be a member of  \emph{ AWS IAM Groups  } . It may be granted direct permissions or indirect permissions via  \emph{ AWS IAM Group  } membership.  \paragraph{} The  \emph{ AWS Account Root User  } is not considered as an  \emph{ AWS IAM User  } .  \end{DIC_Def} \subsection*{Related Terms } \begin{enumerate} \item  ARN  \item  AWS  \item  AWS Account  \item  (link) \href{AWS Account Root User (Dictionary Entry) }{ }   \item  AWS IAM  \item  (link) \href{AWS IAM Group (Dictionary Entry) }{ }   \end{enumerate} \subsection*{Quotes } \begin{DIC_BlockQuote} IAM UserAn AWS Identity and Access Management (IAM)~user~is an entity that you create in AWS to represent the person or application that uses it to interact with AWS. A user in AWS consists of a name and credentials.  \end{DIC_BlockQuote} (AWS, 11/2020, p. 74)\\ (Online: \url{https://docs.aws.amazon.com/IAM/latest/UserGuide/id_users.html})  \paragraph{} (  (link) \href{AWS, 11/2020 }{ } , p. 74)  (Online:  https://docs.aws.amazon.com/IAM/latest/UserGuide/id\_users.html  )  \subsection*{Bibliography } \begin{enumerate} \item  (link) \href{AWS, 11/2020 }{ }   \end{enumerate} \subsection*{See Also } false  title  label = "aws-iam-user"   
 
 % Page Space: DIC 
 % Page ID: 998244664 
 % Page Title: Mutual Authentication (Dictionary Entry) 
 \newpage \subsection*{Mutual Authentication } \subsection*{Definitions } \begin{DIC_Def}{Definition 1 }{} \paragraph{} A communication scheme where both communicating entities are authenticated to each other.  \paragraph{} Mutual authentication requires more than two unilateral authentications in opposite directions, because of the relationship between these two opposite processes.  \paragraph{} Mutual authentication protects against unauthorized access by mitigating man-in-the-middle attacks. In certain circumstances, it may mitigate DoS attacks as well.  \paragraph{} When communication takes place between a server and a client, authentication of the client by the server may be incorrectly perceived as the only important security aspect. But without authentication of the server by the client, the server itself may be spoofed leading the way to multiple attacks.  \end{DIC_Def} \subsection*{Related Terms } \begin{enumerate} \item  (link) \href{Authentication (Dictionary Entry) }{ }   \item  Unilateral Authentication  \end{enumerate} \subsection*{Quotes } \begin{DIC_BlockQuote} SRP-8REQUIREMENT: The CSP SHALL ensure that all communications occur over a mutually authenticated protected channel. (5.3.3.2 \#7)SUPPLEMENTAL GUIDANCE: Mutually authenticated protected channels employ approved cryptography to encrypt communications between (sic)Supervised remote identity proofing stations/kiosks are required to employ mutual authentication where both the station/kiosk and server authenticate to each other. This is most often accomplished through the use of mutual TLS. Upon successful mutual authentication, an encrypted communication channel is established between the workstation/kiosk and the server which protects the data exchanged between them.ASSESSMENT OBJECTIVE: Confirm the CSP's supervised remote identity proofing stations or kiosks communicate with the identity service via mutually authenticated protected channels.POTENTIAL ASSESSMENT METHODS AND OBJECTS: Examine: one or both the of the following:● system documentation, such as remote identity proofing station specifications; or● an actual supervised remote identity proofing station (kiosk) employed by the CSP.  \end{DIC_BlockQuote} (Fenton, 2020, p. 58-59)  \paragraph{} (  (link) \href{Fenton, 2020 }{ } , p. 58-59)  \begin{DIC_BlockQuote} 3.2.2.4 Authentication and Data Integrity between ABAC ComponentsThe authorization service requires strong mutual authentication between ABAC components (e.g., PEP, PDP) when authorization service components exchange sensitive information. For each exchange, proof of origin, data integrity, and timeliness should be considered. For example, when the authorization service needs to obtain attributes from an authoritative attribute service, mutual authentication should be used, followed by mechanisms for validating message integrity and message origin. Authentication protocols based on strong methods (e.g., X.509 authentication) should be used to provide the level of assurance needed by both parties involved in the attribute exchange.  \end{DIC_BlockQuote} (NIST SP 800-162, 2014, p. 28)  \paragraph{} (  (link) \href{NIST SP 800-162, 2014 }{ } , p. 28)  \begin{DIC_BlockQuote} RADIUS(\ldots)- Mutual authentication support: Man-in-the-middle attacks are possible with one-way authentication. Mutual authentication eliminates this risk by authenticating the RADIUS server and the client. The client initially passes its identification to the server, which responds with its identification so that both the server and the client are assured of mutual reliability. The same happens with the AP and the server.  \end{DIC_BlockQuote} (EC-Council, 2010, § 5-35)  \paragraph{} (  (link) \href{EC-Council, 2010 }{ } , § 5-35)  \begin{DIC_BlockQuote} DHCP Services(\ldots)RFC 3118 appends authentication to DHCP and permits a client to confirm whether a specific DHCP server can be relied on and whether a request for DHCP information originates from a client that is certified to use the network. This mutual authentication in DHCP presents the additional security advantage of helping to protect DHCP clients and servers from DoS attacks and unauthorized access. RFC 3118 describes a method that can present both individual certification and message confirmation. This helps a DHCP client verify the uniqueness of the DHCP server it chooses in an unsecured network environment. This operation is very helpful for both a standard company Ethernet network and an Internet service provider (ISP).  \end{DIC_BlockQuote} (EC-Council, 2010, § 5-38-39)  \paragraph{} (  (link) \href{EC-Council, 2010 }{ } , § 5-38-39)  \begin{DIC_BlockQuote} 11.4.2 Mutual AuthenticationThe basic mechanisms for message freshness or principal-liveness introduced so far achieve so-called "unilateral authentication" which means that only one of the two protocol participants is authenticated. In mutual authentication, both communicating entities are authenticated to each other.ISO and IEC have standardized a number of mechanisms for mutual authentication. A signature based mechanism named "ISO Public Key Three-Pass Mutual Authentication Protocol" {[}148{]} is specified in prot 11.1. We choose to specify this mechanism in order to expose a common misunderstanding on mutual authentication.One might want to consider that mutual authentication is simply twice unilateral authentication; that is, mutual authentication could be achieved by applying one of the basic unilateral authentication protocols in §11.4.1 twice in the opposite directions. However, this is not generally true!A subtle relationship between mutual authentication and unilateral authentication was not clearly understood in an early stage of the ISO/IEC standardization process for prot 11.1. (\ldots)  \end{DIC_BlockQuote} (Mao, 2003, § 11.4.2)  \paragraph{} (  (link) \href{Mao, 2003 }{ } , § 11.4.2)  \begin{DIC_BlockQuote} mutual authenticationAuthentication of both ends of a communication session.OverviewTraditional network authentication systems have centered around having the server authenticate the credentials of the client. They ignore authentication of the server by the client since it is assumed that the server is always a trusted entity. However, it is sometimes possible to spoof the identity of a server, especially in an Internet scenario in which information is sent over an insecure public communication system and is subject to eavesdropping, interception, and hijacking. Although simple consumer transactions such as users buying goods online may suffice with one-way authentication of clients by e-commerce servers, more costly business-to-business (B2B) and financial industry transactions need both ends of a communication channel to be authenticated before establishing a session and per- forming a transaction. Mutual authentication is the general term for any scheme by which both parties authenticate the other prior to sending sensitive information to each other.One protocol that was developed for mutual authentication is Kerberos, a popular authentication protocol developed by the Massachusetts Institute of Technology (MIT) and used by Active Directory directory service in Microsoft Windows 2000 and Windows Server 2003. Other mutual authentication protocols include the following:● Microsoft Challenge Handshake Authentication Protocol version 2 (MS-CHAPv2)● Extensible Authentication Protocol/Transport Layer Security (EAP/TLS)● Symmetric-Key Three-Pass Mutual Authentication Protocol defined in the ISO 9798 standardSee Also: authentication, Kerberos  \end{DIC_BlockQuote} (Tulloch, 2003, p. 199)  \paragraph{} (  (link) \href{Tulloch, 2003 }{ } , p. 199)  \subsection*{Bibliography } \begin{enumerate} \item  (link) \href{EC-Council, 2010 }{ }   \item  (link) \href{Fenton, 2020 }{ }   \item  (link) \href{Mao, 2003 }{ }   \item  (link) \href{NIST SP 800-162, 2014 }{ }   \item  (link) \href{Tulloch, 2003 }{ }   \end{enumerate} \subsection*{See Also }  
 
 % Page Space: DIC 
 % Page ID: 998244645 
 % Page Title: Stability of Access Decision Factors (Dictionary Entry) 
 \section*{Stability of Access Decision Factors }\subsection*{Definitions } \begin{DIC_Def}{Definition 1 }{}The average period during which access decision factors are only subject to slight disturbance, prolonging the validity of previously defined access permissions. A disturbance of access decision factors beyond some threshold triggers the requirement to adapt access permissions. Distinct access control methods (e.g. ACL, RBAC, ABAC, PBAC) are varyingly efficient in the way they enable modifications of access permissions. \end{DIC_Def}\subsection*{Related Terms }\begin{itemize} \item   ABAC \item   Access (Dictionary Entry) \item   Access Control (Dictionary Entry) \item   Access Control List (Dictionary Entry) \item   PBAC \item   RBAC \end{itemize} \subsection*{Quotes }\begin{quote} 3.1 \emph{Stability of Access Decision Factors} -- When the basis for access decisions is relatively stable, use of mechanisms such as ACLs lends itself more readily. Administrative processes typically required to maintain these lists are time-intensive and not particularly well suited to situations where significant changes and updates are required frequently. On the other hand, use of a flexible Attribute Management enterprise service where attributes can be easily managed, may be more responsive and thus, more operationally effective. \end{quote}  ( Farroha and Farroha, 2012 , p. 3) \subsection*{Bibliography }\begin{itemize} \item   Farroha and Farroha, 2012 \end{itemize} \subsection*{See Also }false title label = "stability-of-access-decision-factors"  
 
 % Page Space: DIC 
 % Page ID: 998244454 
 % Page Title: Data Custodian (Dictionary Entry) 
 \input{confluence-page-latex/DIC_998244454} 
 
 % Page Space: DIC 
 % Page ID: 998179040 
 % Page Title: Data Owner (Dictionary Entry) 
 \input{confluence-page-latex/DIC_998179040} 
 
 % Page Space: DIC 
 % Page ID: 998178905 
 % Page Title: Authority (Dictionary Entry) 
 \input{confluence-page-latex/DIC_998178905} 
 
 % Page Space: BIB 
 % Page ID: 998146166 
 % Page Title: Milgate, 2006 
 \input{confluence-page-latex/BIB_998146166} 
 
 % Page Space: DIC 
 % Page ID: 998113409 
 % Page Title: Information Owner (Dictionary Entry) 
 \input{confluence-page-latex/DIC_998113409} 
 
 % Page Space: DIC 
 % Page ID: 974454827 
 % Page Title: Concurrent Impersonation Attack (Dictionary Entry) 
 \input{confluence-page-latex/DIC_974454827} 
 
 % Page Space: DIC 
 % Page ID: 974454812 
 % Page Title: Active Impersonation Attack (Dictionary Entry) 
 \input{confluence-page-latex/DIC_974454812} 
 
 % Page Space: DIC 
 % Page ID: 974323764 
 % Page Title: Passive Impersonation Attack (Dictionary Entry) 
 \input{confluence-page-latex/DIC_974323764} 
 
 % Page Space: DIC 
 % Page ID: 974323719 
 % Page Title: Impersonation (Dictionary Entry) 
 \input{confluence-page-latex/DIC_974323719} 
 
 % Page Space: DIC 
 % Page ID: 966492169 
 % Page Title: Authenticated Users (Dictionary Entry) 
 \input{confluence-page-latex/DIC_966492169} 
 
 % Page Space: DIC 
 % Page ID: 881197061 
 % Page Title: Session (Dictionary Entry) 
 \input{confluence-page-latex/DIC_881197061} 
 
 % Page Space: DIC 
 % Page ID: 877297665 
 % Page Title: Continuous Authentication (Dictionary Entry) 
 \input{confluence-page-latex/DIC_877297665} 
 
 % Page Space: DIC 
 % Page ID: 873201665 
 % Page Title: Re-authentication (Dictionary Entry) 
 \input{confluence-page-latex/DIC_873201665} 
 
 % Page Space: DIC 
 % Page ID: 86442069 
 % Page Title: Role (Dictionary Entry) 
 \input{confluence-page-latex/DIC_86442069} 
 
 % Page Space: DIC 
 % Page ID: 858522094 
 % Page Title: Identity Proofing (Dictionary Entry) 
 \input{confluence-page-latex/DIC_858522094} 
 
 % Page Space: DIC 
 % Page ID: 858521682 
 % Page Title: Rejoiner (Dictionary Entry) 
 \input{confluence-page-latex/DIC_858521682} 
 
 % Page Space: DIC 
 % Page ID: 858521641 
 % Page Title: Joiner Process (Dictionary Entry) 
 \input{confluence-page-latex/DIC_858521641} 
 
 % Page Space: DIC 
 % Page ID: 858488984 
 % Page Title: JML (Dictionary Entry) 
 \input{confluence-page-latex/DIC_858488984} 
 
 % Page Space: DIC 
 % Page ID: 858456203 
 % Page Title: Joiner Hire Date (Dictionary Entry) 
 \input{confluence-page-latex/DIC_858456203} 
 
 % Page Space: DIC 
 % Page ID: 858456124 
 % Page Title: New Joiner (Dictionary Entry) 
 \input{confluence-page-latex/DIC_858456124} 
 
 % Page Space: DIC 
 % Page ID: 855506953 
 % Page Title: Joiner (Dictionary Entry) 
 \input{confluence-page-latex/DIC_855506953} 
 
 % Page Space: DIC 
 % Page ID: 852492379 
 % Page Title: Joiner Lead Time (Dictionary Entry) 
 \input{confluence-page-latex/DIC_852492379} 
 
 % Page Space: DIC 
 % Page ID: 852492338 
 % Page Title: Joiner Start Date (Dictionary Entry) 
 \input{confluence-page-latex/DIC_852492338} 
 
 % Page Space: DIC 
 % Page ID: 848724060 
 % Page Title: Authenticator (Dictionary Entry) 
 \input{confluence-page-latex/DIC_848724060} 
 
 % Page Space: DIC 
 % Page ID: 848625750 
 % Page Title: Applicant (Dictionary Entry) 
 \newpage \subsection*{Applicant } \subsection*{Definitions } \begin{DIC_Def}{Definition 1IAMWorkforce IAMIdentity Management }{} \paragraph{} A person going through the identity proofing and onboarding process.  \end{DIC_Def} \subsection*{Related Terms } \begin{enumerate} \item  Enrollement  \item  (link) \href{Entity (Dictionary Entry) }{ }   \item  (link) \href{Identity Proofing (Dictionary Entry) }{ }   \item  (link) \href{Joiner Process (Dictionary Entry) }{ }   \item  Onboarding  \end{enumerate} \subsection*{Quotes } \begin{DIC_BlockQuote} ApplicantA subject undergoing the processes of enrollment and identity proofing.  \end{DIC_BlockQuote} (NIST SP 800-63-3-R3, 2020, p. 39)  \paragraph{} (  (link) \href{NIST SP 800-63-3-R3, 2020 }{ } , p. 39)  \subsection*{Bibliography } \begin{enumerate} \item  (link) \href{NIST SP 800-63-3-R3, 2020 }{ }   \end{enumerate} \subsection*{See Also }  
 
 % Page Space: DIC 
 % Page ID: 848363630 
 % Page Title: Claimant (Dictionary Entry) 
 \input{confluence-page-latex/DIC_848363630} 
 
 % Page Space: DIC 
 % Page ID: 834044330 
 % Page Title: Virtual Directory (Dictionary Entry) 
 \input{confluence-page-latex/DIC_834044330} 
 
 % Page Space: DIC 
 % Page ID: 834044288 
 % Page Title: Metadirectory (Dictionary Entry) 
 \input{confluence-page-latex/DIC_834044288} 
 
 % Page Space: DIC 
 % Page ID: 82903984 
 % Page Title: CSP (Dictionary Entry) 
 \input{confluence-page-latex/DIC_82903984} 
 
 % Page Space: DIC 
 % Page ID: 82903928 
 % Page Title: RA (Dictionary Entry) 
 \input{confluence-page-latex/DIC_82903928} 
 
 % Page Space: DIC 
 % Page ID: 82903896 
 % Page Title: e-Identity (Dictionary Entry) 
 \input{confluence-page-latex/DIC_82903896} 
 
 % Page Space: DIC 
 % Page ID: 82903488 
 % Page Title: RP (Dictionary Entry) 
 \input{confluence-page-latex/DIC_82903488} 
 
 % Page Space: DIC 
 % Page ID: 82903275 
 % Page Title: Identity Attribute (Dictionary Entry) 
 \input{confluence-page-latex/DIC_82903275} 
 
 % Page Space: DIC 
 % Page ID: 82870985 
 % Page Title: Registration Authority (Dictionary Entry) 
 \input{confluence-page-latex/DIC_82870985} 
 
 % Page Space: DIC 
 % Page ID: 82870977 
 % Page Title: Credential Service Provider (Dictionary Entry) 
 \newpage \subsection*{Credential Service Provider }  
 
 % Page Space: DIC 
 % Page ID: 82870848 
 % Page Title: Extended Enterprise (Dictionary Entry) 
 \input{confluence-page-latex/DIC_82870848} 
 
 % Page Space: DIC 
 % Page ID: 82839586 
 % Page Title: User (Dictionary Entry) 
 \input{confluence-page-latex/DIC_82839586} 
 
 % Page Space: DIC 
 % Page ID: 82838735 
 % Page Title: IAM Management (Dictionary Entry) 
 \input{confluence-page-latex/DIC_82838735} 
 
 % Page Space: DIC 
 % Page ID: 82838169 
 % Page Title: Control Party (Dictionary Entry) 
 \input{confluence-page-latex/DIC_82838169} 
 
 % Page Space: DIC 
 % Page ID: 82838020 
 % Page Title: Identity Provider (Dictionary Entry) 
 \input{confluence-page-latex/DIC_82838020} 
 
%\subsection*{Access Control List } 
\begin{enumerate}
	\item  \paragraph{} Ackle  \texttt{ Prononciation  }
	\item  \paragraph{} ACL  \texttt{ Acronym  }
\end{enumerate} 
%\newpage \subsection*{Access Control List } \subsection*{Alternate Forms } \begin{enumerate} \item  Ackle  \texttt{ Prononciation  } \item  ACL  \texttt{ Acronym  } \end{enumerate} \subsection*{Definitions } \begin{DIC_Def}{Definition 1 }{} \paragraph{} A digital representation listing the principals that have access to a resource and the operations that they are authorized to execute on it.  \paragraph{} It is used by the  \emph{ reference monitor  } to allow or deny access requests to the resource.  \paragraph{} It is a  \emph{ discretionary access control  } mechanism, i.e. authorized users such as  \emph{ resource owners  } have the possibility to modify it, effectively granting and revoking access permissions.  \paragraph{} It is linked to (and sometimes embedded in) the resource. This may be an advantage as it provides flexibility with an  \emph{ access granularity level  } set at the individual resource. This may be a disadvantage as managing ACLs at scale becomes inefficient, function of the number of  \emph{ resources  } , the number of  \emph{ principals  } and the  \emph{ stability of access decision factors  } .  \paragraph{} It may be considered as resource metadata.  \end{DIC_Def} \subsection*{Related Terms } \begin{enumerate} \item  (link) \href{Access (Dictionary Entry) }{ }   \item  (link) \href{Access Control (Dictionary Entry) }{ }   \item  (link) \href{Access Granularity (Dictionary Entry) }{ }   \item  (link) \href{AWS ACL (Dictionary Entry) }{ }   \texttt{ Product-specific Implementation  } \item  Discretionary Access Control  \texttt{ Generic Form  } \item  Linux ACL  \item  Resource  \item  (link) \href{Stability of Access Decision Factors (Dictionary Entry) }{ }   \item  Windows ACL  \end{enumerate} \subsection*{Quotes } \begin{DIC_BlockQuote} Access Control List (ACL). The access matrix is implemented through a set of lists, one for each object (i.e., the columns of the matrix) in the system. The list associated with an object has an element for each subject holding a privilege on the object. This element contains the set of privileges the subject can exercise on the object. This is the way usually adopted by modern operating systems.  \end{DIC_BlockQuote} (Ferrari, 2010, p. 12)  \paragraph{} (  (link) \href{Ferrari, 2010 }{ } , p. 12)  \begin{DIC_BlockQuote} 4.2.2 Access Control ListsAnother way of simplifying the management of access rights is to store the access control matrix a column at a time, along with the resource to which the column refers. This is called an access control list or ACL (pronounced `ackle'). In the first of our above examples, the ACL for file 3 (the account file) might look as shown here in Figure 4.4.ACLs have a number of advantages and disadvantages as a means of managing security state. These can be divided into general properties of ACLs, and specific properties of particular implementations.ACLs are a natural choice in environments where users manage their own file security, and became widespread in the Unix systems common in universities and science labs from the 1970s. They are the basic access control mechanism in Unix-based systems such as GNU/Linux and Apple's OS/X; the access controls in Windows are also based on ACLs, but have become more complex over time. Where access control policy is set centrally, ACLs are suited to environments where protection is data-oriented; they are less suited where the user population is large and constantly changing, or where users want to be able to delegate their authority to run a particular program to another user for some set period of time. ACLs are simple to implement, but are not efficient as a means of doing security checking at runtime, as the typical operating system knows which user is running a particular program, rather than what files it has been authorized to access since it was invoked. The operating system must either check the ACL at each file access, or keep track of the active access rights in some other way.Finally, distributing the access rules into ACLs means that it can be tedious to find all the files to which a user has access. Revoking the access of an employee who has just been fired will usually have to be done by cancelling their password or other authentication mechanism. It may also be tedious to run system-wide checks; for example, verifying that no files have been left world-writable could involve checking ACLs on millions of user files.  \end{DIC_BlockQuote} (Anderson, 2008, p. 99)  \paragraph{} (  (link) \href{Anderson, 2008 }{ } , p. 99)  \begin{DIC_BlockQuote} Access control listA list of principals that are authorized to have access to some object.  \end{DIC_BlockQuote} (Saltzer and Schroeder, 1975, p. 1)  \paragraph{} (  (link) \href{Saltzer and Schroeder, 1975 }{ } , p. 1)  \subsection*{Bibliography } \begin{enumerate} \item  (link) \href{Saltzer and Schroeder, 1975 }{ }   \end{enumerate} \subsection*{See Also } 
%\newpage \subsection*{Unanticipated User } \subsection*{Definitions } \begin{DIC_Def}{Definition 1 }{} \paragraph{} A user whose onboarding was not anticipated.  \paragraph{} Unanticipated users may occur when the onboarding process is not established and followed, or when the circumstances that trigger the onboarding process are such that it couldn't be followed.  \paragraph{} The absence of a process to manage the unanticipated users may have adverse effects on the organization. When the onboarding process is not established or followed, it is a managerial issue. When the onboarding process couldn't be followed, depending on requirements, self-registration, identity federation, ABAC, PBAC may help manage  \emph{ unanticipated users  } .  \end{DIC_Def} \subsection*{Related Terms } \begin{enumerate} \item  ABAC  \item  (link) \href{Entity (Dictionary Entry) }{ }   \item  Identity Federation  \item  Onboarding Process  \item  PBAC  \item  Unanticipated Entity  \texttt{ Generic Form  } \item  (link) \href{User (Dictionary Entry) }{ }   \end{enumerate} \subsection*{Quotes } \begin{DIC_BlockQuote} 3.3 Need to Support Unanticipated Users -- The approach for establishing a requesters' identity may be driven by the need to support entities that were not necessarily expected to require such access. For example, in a military operation, there may be a need to expand the involvement of personnel from other agencies e.g., intelligence analysts who were not initially anticipated. If the identity approach selected uses DoD credentials, each analyst identified initially would be issued a DoD credential. In this scenario, each new analyst identified would need to be issued a DoD credential. This would mean that each new analyst has to physically visit a DoD Registration Authority. That operator has to validate that the user's registration is approved, establish the user's true identity, registered him in a DoD repository of authorized users, and create and issue the user a PKI certificate.The requester identity approach selected may be very appropriate for large user populations where users can be identified well in advance of their need for access. However, even if the approval, registration and issuance process could be expedited, the time required to register new personnel may have an adverse impact on the mission operation. It may be more effective to select an identification scheme that can recognize and authenticate identity credentials issued by other US federal agencies. Access control mechanisms such as ABAC and PBAC lend themselves to more sophisticated access control rules that can include provisions for allowing more flexible identification schemes  \end{DIC_BlockQuote} (Farroha and Farroha, 2012, p. 3)  \paragraph{} (  (link) \href{Farroha and Farroha, 2012 }{ } , p. 3)  \subsection*{Bibliography } \begin{enumerate} \item  (link) \href{Farroha and Farroha, 2012 }{ }   \end{enumerate} \subsection*{See Also } 
%\section*{Stability of Access Decision Factors }\subsection*{Definitions } \begin{DIC_Def}{Definition 1 }{}The average period during which access decision factors are only subject to slight disturbance, prolonging the validity of previously defined access permissions. A disturbance of access decision factors beyond some threshold triggers the requirement to adapt access permissions. Distinct access control methods (e.g. ACL, RBAC, ABAC, PBAC) are varyingly efficient in the way they enable modifications of access permissions. \end{DIC_Def}\subsection*{Related Terms }\begin{itemize} \item   ABAC \item   Access (Dictionary Entry) \item   Access Control (Dictionary Entry) \item   Access Control List (Dictionary Entry) \item   PBAC \item   RBAC \end{itemize} \subsection*{Quotes }\begin{quote} 3.1 \emph{Stability of Access Decision Factors} -- When the basis for access decisions is relatively stable, use of mechanisms such as ACLs lends itself more readily. Administrative processes typically required to maintain these lists are time-intensive and not particularly well suited to situations where significant changes and updates are required frequently. On the other hand, use of a flexible Attribute Management enterprise service where attributes can be easily managed, may be more responsive and thus, more operationally effective. \end{quote}  ( Farroha and Farroha, 2012 , p. 3) \subsection*{Bibliography }\begin{itemize} \item   Farroha and Farroha, 2012 \end{itemize} \subsection*{See Also }false title label = "stability-of-access-decision-factors" 
	
	Book end \\
	
\end{document}